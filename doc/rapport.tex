\documentclass[a4paper]{scrartcl}
\usepackage[utf8]{inputenc}
\usepackage[frenchb]{babel}
\usepackage{amsmath} % math
\usepackage{amssymb} % math
\usepackage{gensymb} % math
\usepackage[T1]{fontenc}
\usepackage{lmodern}
\usepackage{graphicx}
\usepackage{url}
    \urlstyle{sf}
\usepackage[usenames]{color}
\usepackage{array}
%\usepackage{etex} % chimie
%\usepackage{m-pictex} % chimie
%\usepackage{m-ch-en} % chimie
%\usepackage{lipsum}% pour mettre du texte aléatoire via \lipsum
\definecolor{codeBlue}{rgb}{0,0,1}
\definecolor{webred}{rgb}{0.5,0,0}
\definecolor{codeGreen}{rgb}{0,0.5,0}
\definecolor{codeGrey}{rgb}{0.6,0.6,0.6}
\definecolor{webdarkblue}{rgb}{0,0,0.4}
\definecolor{webgreen}{rgb}{0,0.3,0}
\definecolor{webblue}{rgb}{0,0,0.8}
\definecolor{orange}{rgb}{0.7,0.1,0.1}
\usepackage{caption}
\renewcommand{\familydefault}{\sfdefault}
\date{\today} %\today
%\institute{Cairo-Dock}
\usepackage{listings}        % Pour l'insersion de fichiers de codes sources.
\lstset{
      language=C,
      flexiblecolumns=true,
      numbers=left,
      stepnumber=1,
      numberstyle=\ttfamily\tiny,
      keywordstyle=\ttfamily\textcolor{blue},
      stringstyle=\ttfamily\textcolor{red},
      commentstyle=\ttfamily\textcolor{green},
      breaklines=true,
      extendedchars=true,
      basicstyle=\ttfamily\scriptsize,
      showstringspaces=false
    }
\title{\texttt{LING1113}: Projet 1 : Multiplication de matrices creuses}
\author{\textsc{Matthieu Baerts} \& \textsc{Hélène Verhaeghe}\\Groupe 21}
\begin{document}
\maketitle
% \tableofcontents

\section{Performances du programme}
Afin de comparer les performances de notre programme avec une implémentation triviale de matrices pleines (avec un tableau à deux dimensions), nous fournissons avec notre code les deux implémentations. Le premier se trouve dans \texttt{matrix.c} et le deuxième dans \texttt{matrix\_plain.c}. Concernant les exécutables, lors de l'utilisation de la commande \texttt{make}, deux binaires seront produits: \texttt{matrixprod} et \texttt{matrixprod\_plain}.

\subsection{Mesures réalisées}
Voici plusieurs mesures réalisées avec plusieurs matrices allant de 90 à 100 éléments par colonnes et lignes. Dans ces tableaux, nous retrouvons à chaque fois, dans la première colonne, le temps d'exécution pour l'implémentation des matrices creuses et dans la seconde, celle pour les matrices pleines.

\begin{table}[H]
    \caption{Creuses à environ 90\%}
    \label{tab:10}
    \begin{center}
        \begin{tabular}{cc}
             & \\
             & \\
        \end{tabular}
    \end{center}
\end{table}

\begin{table}[H]
    \caption{Creuses à environ 95\%}
    \label{tab:}
    \begin{center}
        \begin{tabular}{cc}
             & \\
             & \\
        \end{tabular}
    \end{center}
\end{table}

\begin{table}[H]
    \caption{Creuses à environ 99\%}
    \label{tab:}
    \begin{center}
        \begin{tabular}{cc}
             & \\
             & \\
        \end{tabular}
    \end{center}
\end{table}

\begin{figure}[H]
    \begin{center}
        %\includegraphics{}
        \caption{Représentation graphique des données venant des 3 tableaux précédents}
        \label{fig:time}
    \end{center}
\end{figure}
La représentation graphique se trouve dans la figure \ref{fig:time}.

\section{Utilité de l'implémentation de matrices creuses}

\end{document}
